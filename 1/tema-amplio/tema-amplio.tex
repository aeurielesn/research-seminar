\documentclass[letterpaper,10pt]{article}
\usepackage[T1]{fontenc}
\usepackage{lmodern}
\usepackage{natbib}
%\setlength{\parskip}{\medskipamount}
\setlength{\parindent}{0pt}
\usepackage[spanish]{babel}
%\usepackage[tex4ht]{hyperref} % activate when creating an HTML
\usepackage{hyperref}
%\usepackage{tex4ht} % activate when creating an HTML
\bibliographystyle{unsrt} 

\title{Tema Amplio de Investigaci�n}

\author{Alexander Enrique Urieles Nieto\\
\href{mailto:aeurielesn@gmail.com}{\nolinkurl{aeurielesn@unal.edu.co}}}

\date{Maestr�a en Sistemas y Computaci�n\\
Departamento de Ingenier�a de Sistemas e Industrial\\
Universidad Nacional de Colombia}

\begin{document}

\maketitle

\section*{Tema Amplio de Investigaci�n}

De acuerdo con mi Inter�s de Investigaci�n \citep{Urieles2011a}, 
el �rea de Compresi�n de Informaci�n es mi �rea de inter�s general de 
investigaci�n, el cual hace parte del estudio de las Ciencias de la Computaci�n
y de la Teor�a de la Informaci�n.

A lo largo de los a�os m�ltiples algoritmos han aparecido mejorando
la calidad de la compresi�n especialmente LZ77 \citep{Ziv2002} y LZ78 \citep{Ziv2003},
as� como sus variaciones.

Sin embargo, junto con el aumento en la calidad de la compresi�n, la 
cantidad de informaci�n para almacenar no �nicamente a nivel local sino a nivel mundial
ha aumentado tambi�n, especialmente para las bases de datos. El progreso en las
t�cnicas de compresi�n ha ayudado a aminorar la cantidad de espacio en disco 
que ocupa esta informaci�n.

Desgraciadamente para poder realizar b�squedas en esta informaci�n comprimida
se necesitaba primero descomprimir y utilizar t�cnicas de b�squedas de subcadenas
(como Sunday \citep{sunday1990very}) para encontrar las ocurrencias del patr�n
buscado en el texto comprimido.

Desde el trabajo presentado en \citep{Amir2002} se inici� una nueva rama de investigaci�n
denominada \emph{Compressed Matching Problem}, que busca resolver el problema de 
b�squeda de ocurrencias de un patr�n ya sea exacto, aproximado o complejos directamente
en el texto comprimido sin necesidad de descompresi�n.

En \emph{Compressed Matching Problem} hay una necesidad de encontrar
mejores soluciones para los problemas de ocurrencias exactas, aproximadas y complejas
que sean por lo menos igualmente de r�pidas que en el problema de encontrar las ocurrencias
en el texto descomprimido original.

\bibliography{tema-amplio}

\end{document}
