\documentclass[letterpaper,10pt]{article}
\usepackage[T1]{fontenc}
\usepackage{lmodern}
\usepackage{natbib}
%\setlength{\parskip}{\medskipamount}
\setlength{\parindent}{0pt}
\usepackage[spanish]{babel}
%\usepackage[tex4ht]{hyperref} % activate when creating an HTML
\usepackage{hyperref}
%\usepackage{tex4ht} % activate when creating an HTML
\bibliographystyle{unsrt}

\title{Tema Enfocado de Investigaci�n}

\author{Alexander Enrique Urieles Nieto\\
\href{mailto:aeurielesn@gmail.com}{\nolinkurl{aeurielesn@unal.edu.co}}}

\date{Maestr�a en Sistemas y Computaci�n\\
Departamento de Ingenier�a de Sistemas e Industrial\\
Universidad Nacional de Colombia}

\begin{document}

\maketitle

\section*{Tema Enfocado de Investigaci�n}

Reduciendo el alcance del tema amplio de investigaci�n Compressed Matching Problem
presentado en \cite{Urieles2011d}, que busca resolver el problema de 
b�squeda de ocurrencias de un patr�n ya sea exacto, aproximado o complejos directamente
en el texto comprimido sin necesidad de descompresi�n; escog� como tema enfocado el
\emph{Estudio Comparativo de Algoritmos de B�squeda de Patrones de Cadenas de Caracteres
en Texto Comprimido}.

\bibliography{tema-enfocado}

\end{document}
