\documentclass[letterpaper,10pt]{article}
\usepackage[T1]{fontenc}
\usepackage{lmodern}
%\usepackage[numbers]{natbib}
%\setlength{\parskip}{\medskipamount}
\setlength{\parindent}{0pt}
\usepackage[spanish]{babel}
\usepackage[tex4ht]{hyperref}
%\usepackage{hyperref}
\usepackage{tex4ht} % activate when creating an HTML

\title{Inter�s de Investigaci�n}

\author{Alexander Enrique Urieles Nieto\\
\href{mailto:aeurielesn@gmail.com}{\nolinkurl{aeurielesn@unal.edu.co}}}

\date{Maestr�a en Sistemas y Computaci�n\\
Departamento de Ingenier�a de Sistemas e Industrial\\
Universidad Nacional de Colombia}

\begin{document}

\maketitle

\section*{Inter�s de Investigaci�n}

\subsection*{�rea de Inter�s General}

Dentro del estudio de la Ciencias de la Computaci�n y de la 
Teor�a de la Informaci�n, mi �rea de inter�s general de investigaci�n
es la \textbf{Compresi�n de Datos}.

%Mi tema de inter�s dentro del �rea de \textbf{Compresi�n de Datos} es
%\textbf{B�squedas en Texto Comprimido}.

\subsection*{Tema de Inter�s Espec�fico}

Mi inter�s de investigaci�n en el �rea de Compresi�n de Datos es
estudiar acerca de t�cnicas de \textbf{B�squeda en Texto Comprimido}
para encontrar nuevas aproximaciones, te�ricas o algor�tmicas, al problema
que ayuden a mejorar el rendimiento computacional de �stas.


\end{document}
